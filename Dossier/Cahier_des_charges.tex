\documentclass[a4paper]{article}

\usepackage[utf8]{inputenc}
\usepackage[T1]{fontenc}
\usepackage[francais]{babel}
\usepackage{geometry}

\title{\Huge \textbf{Rapport de projet tuteuré}}
\author{\Large Felton - Gros - Impéras - Piat - Vareille}
\date{15 Janvier 2015}

\begin{document}

\maketitle
\newpage

\tableofcontents
\newpage

\section{Cahier des charges}
	\subsection*{Introduction}
	
	Pour ce deuxième semestre, la réalisation d'un site dynamique est demandée. A l'unanimité, le choix du sujet fût rapidement décidé : un site proposant la vente et la réservation de voiture ainsi que d'accessoires automobiles.
	\subsection{Précision du domaine}
	
	Plusieurs gammes de véhicules seront proposés, selon les besoins, les moyens et les attentes du client. Du véhicule utilitaire à la citadine en passant par des deux roues, le site ne fera aucune exception. Quant au public visé, aucune restriction n'est envisagée. Jeunes conducteurs, routiers ou professionnels recherchant des véhicules appropriés.
	\subsection{Besoins d'utilisateurs}
	Sur le site, l'utilisateur aura accès à plusieurs fonctionnalités, répondants à des besoins particuliers.
	
	En effet, guider la navigation et l'adapter en temps réel à chaque réitération de la recherche est quelque chose de primordial. 
	
	Un aperçu du prix d'une commande via différentes techniques (panier, devis etc.) est une étape importante pour instaurer une confiance de la part du client.
	
	En cas de commande réalisée, un historique précis et détaillé sera disponible à la consultation via un espace personnel propre à l'acheteur.  
	\subsection{Spécifications fonctionnelles}
	Ca marchera pas de toute manière, alors arrête de me faire chier, et va me faire mon MLDR.
	\subsection{Spécifications techniques}
	Même message qu'au dessus.
	\subsection{Spécifications de réalisation}
	La réalisation ? Quelle réalisation ?
	\subsection{Gestion de projet}
	Gestion ? Demandez à Samuel.
	\subsection*{Conclusion}
	Woilà.
\end{document}